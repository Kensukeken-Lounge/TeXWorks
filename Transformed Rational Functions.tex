\documentclass{article}
\usepackage{amsmath}
\usepackage{pgfplots}

\begin{document}

\section{Transformed Rational Functions}

Consider the following two transformed rational functions:

\begin{align*}
f(x) &= \frac{1}{(x-2)^2+1}+3\\
g(x) &= \frac{-2}{x+1}-1
\end{align*}

\section{Key Features}

To inspect the key features of these functions, we can analyze their intercepts, asymptotes, turning points, and increasing/decreasing intervals.

\subsection{$f(x)$}

\begin{itemize}
\item X-intercept: There are no x-intercepts because the denominator is always positive.
\item Y-intercept: When $x=0$, $f(0)=\frac{1}{5}+3=3.2$.
\item Vertical asymptotes: There are no vertical asymptotes.
\item Horizontal asymptote: As $x$ approaches $\pm \infty$, $f(x)$ approaches $3$.
\item Turning points: $f(x)$ has one turning point at $(2,4)$.
\item Increasing/decreasing intervals: $f(x)$ is decreasing on $(-\infty,2)$ and increasing on $(2,\infty)$.
\end{itemize}

\subsection{$g(x)$}

\begin{itemize}
\item X-intercept: When $y=0$, $x=-\frac{1}{2}$.
\item Y-intercept: When $x=0$, $g(0)=-3$.
\item Vertical asymptotes: There is a vertical asymptote at $x=-1$.
\item Horizontal asymptote: As $x$ approaches $\pm \infty$, $g(x)$ approaches $0$.
\item Turning points: $g(x)$ has no turning points.
\item Increasing/decreasing intervals: $g(x)$ is decreasing on $(-\infty,-1)$ and increasing on $(-1,\infty)$.
\end{itemize}

\section{Graphs}

To sketch these functions on a graph, we would need to plot the intercepts, asymptotes, and turning points, and then connect the points smoothly while considering the end behavior and increasing/decreasing intervals. Here are the graphs of $f(x)$ and $g(x)$:

\begin{center}
\begin{tikzpicture}
\begin{axis}[
xlabel=$x$,
ylabel=$y$,
xmin=-5,xmax=5,
ymin=-1,ymax=5,
axis lines=middle,
]
\addplot[domain=-5:5,smooth,blue] {1/((x-2)^2+1)+3};
\addplot[domain=-5:5,smooth,red] {-2/(x+1)-1};
\end{axis}
\end{tikzpicture}
\end{center}

It is important to label the axes properly and use different colors or line styles to distinguish the functions.

\end{document}
