\documentclass{article}
\usepackage{amsmath}

\begin{document}

The Logarithmic Integral Function, denoted by $\text{li}(x)$, is defined as:

\begin{equation*}
\text{li}(x) = \int_{0}^{x} \frac{1}{\ln t} dt.
\end{equation*}

The Logarithmic Integral Function is a special function in number theory and analysis. It is closely related to the prime counting function $\pi(x)$, which counts the number of prime numbers less than or equal to $x$. In fact, the relationship between $\text{li}(x)$ and $\pi(x)$ is given by the Prime Number Theorem, which states that:

\begin{equation*}
\lim_{x \rightarrow \infty} \frac{\text{li}(x)}{x/\ln x} = 1.
\end{equation*}

In other words, the growth rate of $\text{li}(x)$ is similar to that of $x/\ln x$, which is the expected number of primes less than or equal to $x$ according to the Prime Number Theorem.

The Logarithmic Integral Function has several other interesting properties. For example, it is a special case of the more general logarithmic integral:

\begin{equation*}
\text{Li}(x) = \int_{2}^{x} \frac{1}{\ln t} dt,
\end{equation*}

which is used in the study of the Riemann zeta function and other related functions. Another property of $\text{li}(x)$ is that it is a smooth, continuous function for all $x > 0$, and it approaches infinity as $x \rightarrow \infty$. However, it grows very slowly, and for practical purposes it can be considered constant for values of $x$ less than about $10^{10}$.

The Logarithmic Integral Function has numerous applications in mathematics, physics, and engineering. For example, it arises in the analysis of the behavior of various types of functions and systems, and it is used in the calculation of various quantities related to the distribution of primes and other types of numbers.

\end{document}
