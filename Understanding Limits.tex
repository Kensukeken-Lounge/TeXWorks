\documentclass{article}
\usepackage{graphicx} % Required for inserting images

\title{Understanding Limits}
\author{Written by Kensukeken}
\date{February 2023}

\begin{document}

\maketitle

\section{Introduction}
In calculus, the concept of a limit is a fundamental idea that allows us to investigate the behavior of functions as the input values get closer and closer to a certain value. Informally, we say that the limit of a function $f(x)$ as $x$ approaches a certain value $a$ is the value that $f(x)$ gets arbitrarily close to as $x$ gets closer and closer to $a$. This value may or may not be equal to $f(a)$, and the function may or may not be defined at $a$.

To define the limit of $f(x)$ as $x$ approaches $a$, we need to consider the behavior of $f(x)$ for values of $x$ that are arbitrarily close to $a$. Specifically, we say that the limit of $f(x)$ as $x$ approaches $a$ is $L$, denoted by $\lim_{x\to a} f(x) = L$, if for every $\epsilon > 0$ there exists a $\delta > 0$ such that if $0 < |x - a| < \delta$, then $|f(x) - L| < \epsilon$. In other words, we can make $f(x)$ as close as we want to $L$ by making $x$ sufficiently close to (but not equal to) $a$.

One important property of limits is that they are unique, meaning that if the limit of $f(x)$ as $x$ approaches $a$ exists, then it is a unique number. This allows us to use limits to define continuity of functions and to analyze the behavior of functions in various applications, such as optimization problems and modeling real-world phenomena.

Another important concept related to limits is the idea of one-sided limits. A one-sided limit of $f(x)$ as $x$ approaches $a$ from the left, denoted by $\lim_{x\to a^-} f(x)$, is the value that $f(x)$ approaches as $x$ gets closer and closer to $a$ from the left (i.e., for values of $x$ less than $a$). Similarly, a one-sided limit of $f(x)$ as $x$ approaches $a$ from the right, denoted by $\lim_{x\to a^+} f(x)$, is the value that $f(x)$ approaches as $x$ gets closer and closer to $a$ from the right (i.e., for values of $x$ greater than $a$).

To illustrate the concept of limits, let's consider the function $h(x) = \frac{x^2 - 1}{x - 1}$. We want to evaluate the limit of $h(x)$ as $x$ approaches 1. If we simply substitute $x=1$ into $h(x)$, we get an undefined expression. However, we can factor the numerator and simplify to find that $h(x) = x + 1$ for $x\neq 1$. Thus, we can see that $h(x)$ approaches 2 as $x$ approaches 1, and we write:

\[
\lim_{x\to 1} h(x) = 2.
\]

In this example, we can also see that the one-sided limits of $h(x)$ as $x$ approaches 1 from the left and from the right are both equal to 2. This tells us that $h(x)$ is continuous at $x=1$, which means that the function has no jumps or breaks at that point.

In summary, limits are a fundamental concept in calculus and are used to investigate the behavior of functions near certain points. The formal definition of a limit involves the idea of getting arbitrarily close to a certain value, and the key property of limits is that they are unique if they exist. One-sided limits are another important concept, and they can be used to determine whether a function is continuous at a certain point. Overall, limits provide a powerful tool for analyzing the behavior of functions and are used extensively in calculus and other areas of mathematics.

\end{document}
